\documentclass{article}
\usepackage[UTF8]{ctex}
\usepackage[tc]{titlepic}
\usepackage{titlesec}
\usepackage{cite}
\usepackage{fancyhdr}
\usepackage{booktabs}
\usepackage{graphicx}
\usepackage{geometry}
\usepackage[section]{placeins}

\usepackage{amsmath}
\usepackage{cases}
\usepackage{color}
\usepackage{hyperref}
\hypersetup{hypertex=true,
	colorlinks=true,
	linkcolor=blue,
	anchorcolor=blue,
	citecolor=blue}
\usepackage{listings,xcolor}
\usepackage{multirow}

\geometry{a4paper,scale=0.8}
\pagestyle{fancy}

\lhead{第 5 次作业\\\today}
\chead{中国科学技术大学\\数学建模课程}

\rhead{Assignment 5\\ {\CTEXoptions[today=old]\today}}
\newcommand{\upcite}[1]{\textsuperscript{\cite{#1}}}

\titleformat*{\section}{\bfseries\Large}
\titleformat*{\subsection}{\bfseries\large}

\title{\bfseries HW5 title}
\author{name \quad ID}

\begin{document}
	\maketitle
	\begin{abstract}
		
		\par \textbf{关键词}:
	\end{abstract}
	%\clearpage
	% \setcounter{secnumdepth}{1}
	\setcounter{section}{0}
	
	\section{前言(问题的提出)}
	
	\section{问题分析}
	
	\section{建模假设}
	\setcounter{subsection}{0}
    \subsection{坐标建立}
    以地面为z=0平面,风向(与地面平行)为x轴正向,烟囱朝向为z轴正向,烟囱底为坐标原点O,建立O-xyz坐标系;坐标空间离散化,各方向均以1m为一个单位;时间离散化,每0.1s为一个时间单位。
    \subsection{边界条件}
    \begin{enumerate}
    \item 无穷远处烟尘浓度为0;
    \item 烟尘无法穿过地面,即地面对烟尘具有反射作用,体现为z=0时的边界条件,通过反射系数$\alpha$控制。
    \end{enumerate}
    \subsection{额外条件}
    \begin{enumerate}
    \item 初始t=0时,全空间c处处为0;
    \item 仅在烟囱口(0,0,h)处,浓度以恒定速率增加(Q $kg/s$,考虑到离散化后单位体积是1$m^3$,也可以直接认为是Q $kg/(m^3·s$));
    \item 考虑烟尘受重力影响,可以用一个恒定且处处相等、沿z轴负向的风速w来模拟,通过斯托克斯定律估算(\textbf{待补充});
    \item (可选)考虑烟尘从烟囱喷出时向上的初速度,等价于一个在烟囱口附近沿z轴正向的风速。
    \end{enumerate}
	\section{符号说明}
	\begin{table}[htbp]
		\centering
		\begin{tabular}{ccc}
			\hline
			符号 & 含义 & 单位 \\
			\hline
			D & 扩散系数 & $m^2/s$ \\
            $\alpha$ & 地面对烟尘反射系数 &  \\
			Q & 污染物排放速率 & $kg/s$ \\
			v & 风速 & $m/s$ \\
            w & 模拟烟尘重力的风速 & $m/s$\\
            c & 烟尘浓度 & $kg/m^3$ \\
			\hline
		\end{tabular}
	\end{table}
	风向沿x轴正向(以风向为正向建立x轴),模拟烟尘重力的速度沿z轴负向。
	
	\section{数学模型建立}
	\begin{align*}
		&Given ~ D, ~ v, ~ Q, \quad solve ~ c(x,y,z,t),\\
		&s.t.
		\begin{cases}
			\frac{\partial c}{\partial t} = D\Delta c - v\frac{\partial c}{\partial x} - w\frac{\partial c}{\partial z} + tf(x,y,z)	\\
			c|_{x=\pm\infty}=c|_{y=\pm\infty}=c|_{z=+\infty}=0	\\
			\frac{\partial c}{\partial z}|_{z=0} = 0	\\
			c|_{t=0}=0
		\end{cases}\\
		&where  ~
		f(x,y,z) = 
		\begin{cases}
			Q, \quad if~(x, y, z)=(0, 0, h)\\
			0, \quad otherwise
		\end{cases},\\
		&-\infty<x, y<+\infty, \quad 0\leq z, t<+\infty
	\end{align*}
	
	\section{问题求解}
	
	\section{讨论与优化}
	
	\section{组内分工}
	\begin{itemize}
		\item 假设提出与模型建立
		\item 代码编写与绘图仿真
		\item 结果讨论与模型改进
		\item 结果汇总与报告撰写
	\end{itemize}
	
%	\bibliographystyle{ieeetr}
%	\begin{thebibliography}{2}
%		Patrick Pérez, Michel Gangnet, and Andrew Blake. 2003. Poisson image editing. In ACM SIGGRAPH 2003 Papers (SIGGRAPH '03). Association for Computing Machinery, New York, NY, USA, 313–318.
%	\end{thebibliography}
	
\end{document}